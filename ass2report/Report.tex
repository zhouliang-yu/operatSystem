\documentclass{article}
\usepackage[utf8]{inputenc}
\usepackage{amsmath}
\usepackage{color}
\usepackage{graphicx}
\usepackage{cases}
\title{CSC3150 Assignment2}
\author{Yu, Zhouliang 120040077 }
\date{October 2021}

\begin{document}

\maketitle



\section{What I Learnt From the task}
\subsection{-lpthread}
Q: Why gcc does not link to the pthread by "\#include$<$pthread.h$>$", while we must use "-lpthread"
to compile the program
\\
A: 

Having \#include $<$pthread.h$>$ in your code doesn't link in the library; it only includes the header for compilation. That allows the compiler to see the various structures, function declarations, etc. included. Having -lpthread actually causes the linking to be done by the linker. So the include tells the compiler what's available, and the -lpthread actually allows the program to call the functions within the library at runtime.
\subsection{pthread\_join}
Q: Why do we use pthread\_join in multiple thread programming
\\
A: if your main thread(the thread executed by the main), exits before other threads, then it will cause bugs. With pthread\_join, the main thread is waiting until other thread exit.

\end{document}